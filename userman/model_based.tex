\subsection{Model Based Ultrasonic  Array Imaging}

In this section a short introduction to model based ultrasonic imaging is presented. Model based ultrasonic array
imaging~\cite{Stoughton1993,Lingvall2003,Lingvall2007} is different from delay-and-sum imaging in the sense that is
based on optimal information processing  whereas delay-and-sum imaging is based on geometrical focusing. The idea is
to use a (linear) model, taking into account the diffraction effects associated with each transmitter/receiver, the
electrical characteristics for each transmitter/receiver, as well as the used input signal(s), and
then estimate the parameters (the scattering strengths) of the model based on both data and prior
information.
%
The DREAM toolbox can be used in this process for computing the SIRs for the model which can then be convolved
with measured electrical impulse responses to obtain a model for a real measurement setup or using only simulated
impulse responses to evaluate different array designs, for example.

To obtain a linear mode we need to consider both the forward process and the backward process.
As discussed in Section~\ref{sec:analythical} [Eq.~\eqref{eq:pressure_response}] the forward response can be
divided in three parts: the input signal $u(t)$, the forward electro-acoustical response
%
\ifthenelse{\boolean{TEX4HT}}
{
  $h^{ef}(t)$,
}{
  $h^\text{ef}(t)$,
}
and the forward SIR
\ifthenelse{\boolean{TEX4HT}}
{
  $h^{{f-}\smt{sir}}(\r,t)$.
}{
  $h^{\text{f-}\smt{sir}}(\r,t)$.
}
The backward response can similarly be divided in two parts: the
backward acousto-electrical impulse
response
\ifthenelse{\boolean{TEX4HT}}
{
  $h^{eb}(t)$ and the backward SIR $h^{{b-}\smt{sir}}(\r,t)$.
}{
  $h^\text{eb}(t)$ and the backward SIR $h^{\text{b-}\smt{sir}}(\r,t)$.
}
Now, consider an array with $K$ transmit elements and $L$ receive elements and contributions from a single observation
point, $\r = \begin{bmatrix} x &y & z\end{bmatrix}^T$, where ${}^T$ denotes the transpose operator.
The received signal, $y_l(\r,t)$, from the $l$th receive element can be expressed
%%
\ifthenelse{\boolean{TEX4HT}}
{
  \begin{equation}
    \begin{split}
      y_l(\r,t) & =
      \overset{{Forward impulse response (f)}}{\overbrace{\left( \sum_{k=0}^{K-1}
            h^{{f-}\smt{sir}}_k(\r,t) * h_{k}^{ef}(t),
            * u_k(t) \right)}} o(\r) * \\
      & \overset{{Backward impulse response (b)}}{\overbrace{h^{{b-}\smt{sir}}_l(\r,t) *  h_{l}^{eb}(t)}} + e_l(t), \\
      & = h^{f}(\r,t) * h_l^{b}(\r,t) o(\r) + e_l(t), \\
      & = h_l(\r,t) o(\r) + e_l(t),
    \end{split}
    \label{eq:discrete_system_impulse_response}
  \end{equation}
}{
  \begin{equation}
    \begin{split}
      y_l(\r,t) & =
      \overset{\text{Forward impulse response (f)}}{\overbrace{\left( \sum_{k=0}^{K-1}
            h^{\text{f-}\smt{sir}}_k(\r,t) * h_{k}^\text{ef}(t),
            * u_k(t) \right)}} o(\r) * \\
      & \overset{\text{Backward impulse response (b)}}{\overbrace{h^{\text{b-}\smt{sir}}_l(\r,t) *  h_{l}^\text{eb}(t)}} + e_l(t), \\
      & = h^\text{f}(\r,t) * h_l^\text{b}(\r,t) o(\r) + e_l(t), \\
      & = h_l(\r,t) o(\r) + e_l(t),
    \end{split}
    \label{eq:discrete_system_impulse_response}
  \end{equation}
}
%%
where $*$ denotes temporal convolution and $e_l(t)$ is the noise for the $l$th receive element.
Note that the total forward impulse response is a superposition of the forward impulse responses
corresponding to all transmit elements. The \emph{object function}
$o(\r)$ is the scattering strength at the observation point $\r$,
\ifthenelse{\boolean{TEX4HT}}
{
  $h_{k}^{ef}(t)$
}{
  $h_{k}^\text{ef}(t)$
}
is the forward electrical impulse response for the $k$th transmit element,
\ifthenelse{\boolean{TEX4HT}}
{
  $h_{l}^{eb}(t)$
}{
  $h_{l}^\text{eb}(t)$
}
the backward electrical impulse response
for the $l$th receive element, and $u_k(t)$ is the input signal for the $k$th transmit element.
%Hereafter we will refer to the pulse-echo (double-path) impulse $h_l(\r,t)$ as the \emph{system}
%impulse response.

A discrete-time version of~\eqref{eq:system_impulse_response} is obtained by sampling the
impulse responses and by using discrete-time convolutions. If we consider $N$ observation points
then the received discrete waveform from a target at the $n$th observation point,
$\r_n = (x_n,y_n,z_n)$, can be expressed as
%%
\begin{equation}
  \y_l^{(n)} = \h_l^{(n)}(\o)_{n} + \e_l,
  \label{eq:one-pt}
\end{equation}
%%
where the column vector $\h_l^{(n)}$ is the discrete system impulse response for
the $l$th receive element.\footnote{Here all vectors are by convention column vectors.} The vector $\o$
represents the $N$ scattering amplitudes in the region-of-interest, and the notation $(\n)_{n}$ denotes the $n$th
element in $\o$.\footnote{The vector $\o$ can easily be rearrange to form an image when two-dimensional imaging is
  considered~\cite{Lingvall2004}.}
%%

To obtain the received signal for all observation points we need to perform a summation over $n$, which
equivalently can be expressed as a matrix-vector multiplication, according to
%%
\begin{equation}
  \begin{split}
  \y_l = & \sum_{n}\h_l^{(n)}(\o)_{n} + \e_l,\\
  = &\begin{bmatrix} \h_l^{(0)} &  \h_l^{(1)} & \cdots & \h_l^{(N-1)} \end{bmatrix}\o + \e_l,\\
  = & \P_l\o + \e_l,\\
  \end{split}
  \label{eq:all-pts}
\end{equation}
%%
which gives us a liner model for the data for one receive element.

To obtain a model for all elements we can append all $L$ receive signals $\y_l$ into a single vector $\y$ and
we finally have linear model for the total array setup
%%
\begin{equation}
  \begin{split}
    \y = & \begin{bmatrix} \y_{0} \\
      \y_1 \\  \vdots \\
      \y_{L-1}\\  \end{bmatrix}  = \begin{bmatrix} \P_{0} \\
      \P_1 \\  \vdots \\
      \P_{L-1}\\  \end{bmatrix}
    \o +
    \begin{bmatrix} \e_0 \\ \e_1 \\  \vdots \\ \e_{L-1} \\ \end{bmatrix} \\
    = & \P \o + \e,
  \end{split}
  \label{eq:extended}
\end{equation}
%%
for the total array imaging system~\cite{Lingvall2004}.
%
The \emph{propagation matrix}, $\P$, in~\eqref{eq:extended} now
describes both the transmission and the reception process for an arbitrary focused
array. Note that the position of the observation points, and the corresponding scattering amplitudes represented
by the vector $\o$, is not restricted to a regular two-dimensional grid, which is often used in ultrasonic imaging.
%
Furthermore, the array elements can in fact, similar to the observation points , be positioned at arbitrary locations in
tree-dimensional space space. Thus, the model~\eqref{eq:extended} can also
be used to model two-dimensional arrays as well as to model array responses in tree-dimensional space.
Also note that the ``noise'' vector $\e$ describes the \emph{uncertainty} of the model~\eqref{eq:extended}. The noise $\e$
does not only model the measurement noise but also all other errors that we may have, such as: multiple scattering effects,
cross talk between array elements, non-uniform sound speed in the media, etc.

%The DREAM toolbox has a function \texttt{build\_propagation\_matrix} to facilitate computing the propagation
%matrix for a linear array. The function uses a concave linear array transducer function but is should not
%be too difficult for a user to adapt it to other transducer geometries as well.
%After computing the propagation matrix the array system can be simulated by using Eq.~\eqref{eq:extended}.

\subsubsection{The Matched Filter}

The matched filter has the property of maximizing the signal-to-noise ratio (at a single point). The matched filter
for each observation point is given by~\cite{Lingvall2003}
%
\begin{equation}
  \hat{\o}= \P^T\y.
  \label{eq:matched-filter}
\end{equation}

Note that the structure of the matched filter is similar to delay-and-sum processing which  also can be expressed
as a matrix-vector multiplication (see the \texttt{model\_based\_example.m} on the DREAM website),
\begin{equation}
  \hat{\o}= \D^T\y
  \label{eq:das}
\end{equation}
%
where the delay matrix $\D$ has ones in the positions corresponding the the propagation delays and zeros otherwise.

\subsubsection{The Optimal Linear  Estimator}

The optimal linear estimator (or the Wiener filter) is given by~\cite{Stoughton1993,Lingvall2007}
%
\begin{equation}
  \begin{split}
    \hat{\o} & = \Co\P^T(\P\Co\P^T+\Ce)^{-1}\y\\
    & = (\P^T\Ceinv\P+\Coinv)^{-1}\P^T\Ceinv\y,
  \end{split}
\label{eq:optimal}
\end{equation}
%
where $\Co$ is the covariance matrix for $\o$ and $\Ce$ is the covariance matrix for $\e$, respectively.
The estimator~\eqref{eq:optimal} has the property, not found for the matched filter or for delay-and-sum
methods, namely that any beampattern can be compensated given that the signal-to-noise ratio is sufficient.

As a final not on model based imaging is that the matrices involved normally become rather large. It is
therefore highly recommended to used a tuned linear algebra library, such as
\href{http://www.tacc.utexas.edu/resources/software/}{K. Goto's} BLAS library or
the \href{http://math-atlas.sourceforge.net/}{ATLAS} library, for example. These libraries often have thread
support so that all CPUs on the computer can be utilized.

An example of model based (and matrix based delay-and-sum) imaging can be found on the DREAM website (see the
\texttt{model\_base\_example.m} file on the examples page).
